\documentclass[11pt, a4paper]{article}
\usepackage[utf8]{inputenc}
\usepackage[spanish, es-tabla, es-nodecimaldot]{babel}
\usepackage{sansmathfonts}				% Sans Serif equations
\renewcommand*\familydefault{\sfdefault} 		% Sans Serif as default font
\usepackage[a4paper, 					% Page Layout
                     %showframe,				% This shows the frame
                     includehead,
                     footskip=7mm, headsep=6mm, headheight=4.8mm,
                     top=25mm, bottom=25mm, left=25mm, right=25mm]{geometry}
\RequirePackage{caption} 				% Caption customization
\captionsetup{justification=centerlast,font=small,labelfont=sc,margin=1cm}
\usepackage{hyperref}
\hypersetup{
    colorlinks=true,
    linkcolor=blue,
    filecolor=magenta,      
    urlcolor=blue,
    citecolor=blue,    
}
\usepackage{fancyhdr}
\fancyhead{}
\lhead{A la izquierda}
\rhead{\framebox{A la derecha con borde}}
\fancyfoot{}
\rfoot{\thepage}
\pagestyle{fancy}
\renewcommand{\headrulewidth}{0pt}
\renewcommand{\footrulewidth}{0.5pt}

\usepackage{array}
\newcommand{\PreserveBackslash}[1]{\let\temp=\\#1\let\\=\temp}
\newcolumntype{C}[1]{>{\PreserveBackslash\centering}p{#1}}
\newcolumntype{R}[1]{>{\PreserveBackslash\raggedleft}p{#1}}
\newcolumntype{L}[1]{>{\PreserveBackslash\raggedright}p{#1}}

\usepackage{tikz}
\usetikzlibrary{babel}

\usepackage{amsmath}
\usepackage{bm}

\usepackage{amsmath}
\usepackage{bm}

\title{Ecuaciones}
\author{Dr. Ing. Pablo Cossutta}
\date{2019}

\begin{document}
\maketitle
\section{Ecuaciones}
Para utilizar ecuaciones es necesario incluir el paquete amsmath.

Existen 3 formas de escribir ecuaciones, inline, entre \textbackslash \$\dots \$, $E=mc^2$.

Entre \textbackslash \$\$\dots \$\$, $$E=mc^2$$.

O con \textbackslash begin\{equation\} \dots \textbackslash end\{equation\} 
\begin{equation}
	E=mc^2
\end{equation}

Si no se quiere numerar, se utiliza \textbackslash begin\{equation*\} \dots \textbackslash end\{equation*\} 
\begin{equation*}
	E=mc^2
\end{equation*}

\textbackslash begin\{split\} \dots \textbackslash end\{split\}
\begin{equation}
\begin{split}
	a\,x^2+b\,x+c = 0\\
	x = \frac{-b \pm \sqrt{b^2-4ac}}{2a}
\end{split}
\end{equation}

\textbackslash begin\{aligned\} \dots \textbackslash end\{aligned\}
\begin{equation}
\begin{aligned}
	a\,x^2+b\,x+c &= 0\\
	x &= \frac{-b \pm \sqrt{b^2-4ac}}{2a}
\end{aligned}
\end{equation}

Varios
\begin{equation*}
\begin{split}
	\lim_{n \to\infty} x\left(t\right) \\
	\int_0^T x\left(t\right) dt \\
	\iint_V x\left(u,v\right) du\,dv \\
	\sum_{i=0}^n\left(i\right) \\
	\alpha \; \beta \; \gamma \; \Omega \; \mu \bm{v}
\end{split}
\end{equation*}
\end{document}