\documentclass[11pt, a4paper]{article}
\usepackage[utf8]{inputenc}
\usepackage[spanish, es-tabla, es-nodecimaldot]{babel}
\usepackage{sansmathfonts}				% Sans Serif equations
\renewcommand*\familydefault{\sfdefault} 		% Sans Serif as default font
\usepackage[a4paper, 					% Page Layout
                     %showframe,				% This shows the frame
                     includehead,
                     footskip=7mm, headsep=6mm, headheight=4.8mm,
                     top=25mm, bottom=25mm, left=25mm, right=25mm]{geometry}
\RequirePackage{caption} 				% Caption customization
\captionsetup{justification=centerlast,font=small,labelfont=sc,margin=1cm}
\usepackage{hyperref}
\hypersetup{
    colorlinks=true,
    linkcolor=blue,
    filecolor=magenta,      
    urlcolor=blue,
    citecolor=blue,    
}
\usepackage{fancyhdr}
\fancyhead{}
\lhead{A la izquierda}
\rhead{\framebox{A la derecha con borde}}
\fancyfoot{}
\rfoot{\thepage}
\pagestyle{fancy}
\renewcommand{\headrulewidth}{0pt}
\renewcommand{\footrulewidth}{0.5pt}

\usepackage{array}
\newcommand{\PreserveBackslash}[1]{\let\temp=\\#1\let\\=\temp}
\newcolumntype{C}[1]{>{\PreserveBackslash\centering}p{#1}}
\newcolumntype{R}[1]{>{\PreserveBackslash\raggedleft}p{#1}}
\newcolumntype{L}[1]{>{\PreserveBackslash\raggedright}p{#1}}

\usepackage{tikz}
\usetikzlibrary{babel}

\usepackage{amsmath}
\usepackage{bm}

\usepackage{csvsimple}

\begin{filecontents*}{bode.csv}
frec,vin,vout,fase
100,10,10,0
1000,10, 9,1
10000,10,7,-45
\end{filecontents*}

\title{Primer documento}
\author{Dr. Ing. Pablo Cossutta}
\date{2019}

\begin{document}
\maketitle
\section{Tablas}
Las tablas en \LaTeX \space son sumamente configurables, el formato estándar se observa en la Tabla \ref{table1}.
\begin{table}[h!]
	\centering
	\caption{Ejemplo de tabla}
	\label{table1}
	\begin{tabular}{c | c c}
		1 & 2 & 3 \\
		\hline
		4 & 5 & 6\\
		7 & 8 & 9 \\
	\end{tabular}
\end{table}

La Tabla \ref{table2} es la misma que la anterior pero con 	\textbackslash setlength\textbackslash tabcolsep\}\{1pt\}.
\begin{table}[h!]
	\centering
	\caption{Ejemplo de tabla con \textbackslash tabcolsep}
	\label{table2}
	\setlength\tabcolsep{1pt}
	\begin{tabular}{c | c c}
		1 & 2 & 3\\
		\hline
		4 & 5 & 6\\
		7 & 8 & 9 \\
	\end{tabular}
\end{table}

La Tabla \ref{table3} es similar a la Tabla \ref{table1}  con 	\textbackslash renewcommand\{\textbackslash arraystretch\}\{2\}. Lo normal (utilizado en el formato IEEE) es utilizar el valor $1.3$.
\begin{table}[h!]
	\centering
	\caption{Ejemplo de tabla con \textbackslash arraystrech}
	\label{table3}
	\renewcommand{\arraystretch}{2}
	\begin{tabular}{c | c c}
		1 & 2 & 3 \\
		\hline
		4 & 5 & 6\\
		7 & 8 & 9\\
	\end{tabular}
\end{table}

En la Tabla \ref{table4} se forzó un espacio adicional entre dos columnas.
\begin{table}[h!]
	\centering
	\caption{Ejemplo de tabla con un espacio entre dos columnas}
	\label{table4}
	\begin{tabular}{c | c@{\hspace{2cm}} | c}
		1 & 2 & 3 \\
		\hline
		4 & 5 & 6\\
		7 & 8 & 9\\
	\end{tabular}
\end{table}

Si se desea forzar un cambio de página es posible utilizar el comando \textbackslash newpage.
\newpage

Los campos de las tablas comúnmente utilizados son c, r, l y p\{\dots\} y se utilizan para la alineación horizontal. El p incluye el tamaño de la celda. En la Tabla \ref{table5} se utiliza este formato. 
\begin{table}[h!]
	\centering
	\caption{Ejemplo de tabla con columna tipo p}
	\label{table5}
	\begin{tabular}{c | c p{0.3\textwidth}}
		1 & 2 & Este es un texto un poco mas largo que solamente un número\\
		\hline
		4 & 5 & 6\\
		7 & 8 & 9 \\
	\end{tabular}
\end{table}

Si se desea usar l, r o c pero con tamaño fijo pueden utilizarse los comandos L, R y C definidos en el preámbulo, a modo de ejemplo, en la Tabla \ref{table6} se utilizan 3 columnas de tamaño igual a $0.3$ del tamaño del texto.
\begin{table}[h!]
	\centering
	\caption{Ejemplo de tabla centrada con tamaño fijo}
	\label{table6}
	\begin{tabular}{| C{0.3\textwidth} | C{0.3\textwidth} | C{0.3\textwidth} |}
		\hline
		\bfseries Col. 1 & \bfseries Col. 2 & \bfseries Col. 3 \\
		\hline
		1 & 2 & 3\\
		4 & 5 & 6\\
		7 & 8 & 9 \\
		\hline
	\end{tabular}
\end{table}

La Tabla \ref{table7} utiliza los datos proporcionados por el archivo ``bode.csv'' el cual se genera automáticamente con filecontents*.
%\csvautotabular{bode.csv}
%\csvreader[late after line=\\]{bode.csv}{1=\frec,2=\vin,3=\vout,4=\fase}{\thecsvrow & \frec & \vin & \vout & \fase}
\begin{table}[h!]
	\centering
	\caption{Ejemplo de tabla centrada con datos desde ``bode.csv'}
	\label{table7}
	\begin{tabular}{c c c c c}%
		\bfseries \# & $\bm{f}$ & $\bm{v_{in}}$ & $\bm{v_{out}}$ & $\bm{\theta}$ \\ \hline
		\csvreader[head to column names, late after line=\\]{bode.csv}{}{\thecsvrow & \frec & \vin & \vout & \fase}
		\hline
	\end{tabular}
\end{table}

Adentro del \textit{enviroment} tabular se puede poner cualquier cosa (muy útil para organizar un arreglo de gráficos).

\section{Herramientas}
La web \href{https://www.tablesgenerator.com}{LaTeX Tables Generator} permite generar el código correspondiente a una tabla incluyendo datos externos a partir de un archivo CSV.

El paquete \href{https://ctan.org/tex-archive/macros/latex/contrib/booktabs/}{bookstabs} permite gran versatilidad de configuración en las tablas mientras que 
el paquete \href{https://ctan.org/pkg/multirow?lang=en}{multirow} se utiliza para unir varias filas o varias columnas dentro de una tabla.
\end{document}