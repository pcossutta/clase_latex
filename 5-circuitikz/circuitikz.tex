\documentclass[11pt, tikz, multi=page]{standalone}
\usepackage[utf8]{inputenc}
\usepackage{circuitikz}
\usepackage{hyperref}
\hypersetup{
    colorlinks=true,
    linkcolor=blue,
    filecolor=magenta,      
    urlcolor=blue,
    citecolor=blue,    
}

\begin{document}
\begin{page}%
Documentación del paquete \href{https://ctan.org/pkg/circuitikz?lang=en}{circuitikz}.
\end{page}
\begin{page}%
	\begin{circuitikz}[american voltages]
		\tikzstyle{every node}=[font=\footnotesize]	
		\draw
		(0,-1) to[american voltage source, l=$v\left(t\right)$, invert] ++(0,2) -- ++(0.5,0)
		to[R, l=$R$, v=$v_r$, i=$i\left(t\right)$] ++(2,0) -- ++(0.5,0)
		to[C, l=$C$, v=$v_c$] ++(0,-2)
		to[short, -*] ++ (-1.5,0) node[ground, scale=0.6]{} to[short] ++(-1.5,0);
	\end{circuitikz}
\end{page}

\newcommand{\csicircuit}[1] % #1 = name
%
{node(#1_up){}
to[short, *-] ++(-1,0) to[nos, l=$s_1$] ++(0,-1.5) node(#1_a){} --++(0, -1.3) to[nos, l=$s_4$] ++(0,-1.5) to[short, -*] ++(1,0)
(#1_up) to[nos, l=$s_2$] ++(0,-1.5) --++(0, -0.65) node(#1_b){} --++(0, -0.65) to[nos, l=$s_5$] ++(0,-1.5)
(#1_up) to[short] ++(1,0) to[nos, l=$s_3$] ++(0,-1.5) --++(0, -1.3) node(#1_c){} to[nos, l=$s_6$] ++(0,-1.5) --++(-1,0)}
%
%Fig. 1
\begin{page}%
	\begin{circuitikz}[american voltages]
		\tikzstyle{every node}=[font=\footnotesize]	
		\draw
		(0,-2.4) to[short] ++(0,1.4)
		to[american current source, l=$i_{csi}$] ++(0,2) --++(0,1.4)
		to[short] ++(3,0) to [short, -*] ++(0,-0.25)
		\csicircuit{csi} % Expand Macro
		to[short, *-] ++(0, -0.25) --++(-3, -0)
		(1.7,-2.5) to[open, v^<=$v_{csi}$] ++(0,5) % VCSI
		(csi_a) node[left]{$a$} to[short, *-] ++(2.5,0) --++(1,0) node[above]{$i_{inv_a}$} --++(0,0) node[currarrow]{}
		(csi_b) node[left]{$b$} to[short, *-] ++(1.5,0) --++(1,0) node[above]{$i_{inv_b}$} --++(0,0) node[currarrow]{}
		(csi_c) node[left]{$c$} to[short, *-] ++(0.5,0) --++(1,0) node[above]{$i_{inv_c}$} --++(0,0) node[currarrow]{};
	\end{circuitikz}%
\end{page}
\end{document}
